% Beamer presentation: Optimal Execution – From Almgren–Chriss to Deep RL
% Author: <Your Name>
% Date: \today
\documentclass[11pt]{beamer}

% --------------------------
% Beamer setup
% --------------------------
\usetheme{Madrid}
\usecolortheme{seahorse}
\setbeamertemplate{navigation symbols}{}

% Packages
\usepackage{amsmath,amsfonts,amssymb}
\usepackage{graphicx}
\usepackage{booktabs}
\usepackage{hyperref}
\usepackage{multicol}
\usepackage{tikz}

% Custom commands
\newcommand{\todo}[1]{\textcolor{red}{\bf TODO: #1}}
\newcommand{\code}[1]{\texttt{#1}}

% Title information
\title[Optimal Execution via RL]{Optimal Execution – From the Almgren--Chriss Model to Deep Reinforcement Learning}
\author{<Your Name>}
\institute{<Your Institution>}
\date{\today}

% ================================================================
\begin{document}
	
	% ----------------------------------------------------------------
	% Title slide
	% ----------------------------------------------------------------
	\begin{frame}[plain]
		\titlepage
	\end{frame}
	
	% ----------------------------------------------------------------
	% Agenda
	% ----------------------------------------------------------------
	\begin{frame}{Agenda}
		\tableofcontents
	\end{frame}
	
	% ================================================================
	\section{Motivation}
	
	\begin{frame}{Why Optimal Execution Matters}
		\begin{itemize}
			\item Large orders move prices \textit{(market impact)} -- poor scheduling increases costs.
			\item Execution quality is traditionally measured by Implementation Shortfall (IS) and Expected Shortfall (ES).
			\item Classic solution: \alert{Almgren--Chriss (AC)} closed-form optimal schedule under specific assumptions.
			\item But \textbf{market micro-structure is richer}: non-linear impact, stochastic liquidity, fees, etc.
		\end{itemize}
	\end{frame}
	
	\begin{frame}{Why Move Beyond Almgren--Chriss?}
		\begin{columns}[T]
			\column{0.55\textwidth}
			\textbf{Limitations of AC}
			\begin{itemize}
				\item Assumes arithmetic Brownian motion (ABM) or deterministic drift.
				\item Linear temporary and permanent impact.
				\item Risk captured only via variance of proceeds.
				\item Static schedule $\Rightarrow$ cannot react to intra-day price information.
			\end{itemize}
			\column{0.45\textwidth}
			\vspace{-0.3cm}
			\begin{block}{Opportunity for RL}
				\begin{itemize}
					\item Model-free: learn directly from simulated or historical LOB.
					\item Naturally handles high-dimensional states/actions.
					\item Can optimise non-linear objectives (e.g.\ \textit{CVaR}).
				\end{itemize}
			\end{block}
		\end{columns}
	\end{frame}
	
	% ================================================================
	\section{Problem Formulation as an MDP}
	
	\begin{frame}{State Design}
		\textbf{Chosen state $s_k = \big(\underbrace{r_{k-D+1:k}}_{\text{log-returns}}, \underbrace{m_k}_{\text{time left}}, \underbrace{i_k}_{\text{inventory left}}\big)$}
		\begin{itemize}
			\item $r_{k-D+1:k}$: window of $D=5$ past log-returns (6 returns $\Rightarrow$ 6 periods) captures recent momentum and volatility.
			\item $m_k=\tfrac{T-k}{T}$: fraction of horizon remaining provides a natural clock.
			\item $i_k=\tfrac{Q_k}{Q_0}$: remaining inventory fraction informs risk of holding.
		\end{itemize}
		\vspace{0.5em}
		\textbf{Is $D=5$ optimal?}
		\begin{itemize}
			\item Larger $D$ offers richer autocorrelation signals but increases dimension \& sample complexity.
			\item Smaller $D$ reduces noise resilience.
			\item \emph{Empirically}, $D \in [3,8]$ showed marginal gains; \alert{tune via validation}.
		\end{itemize}
	\end{frame}
	
	\begin{frame}{Action Space}
		\begin{itemize}
			\item Continuous \code{a\_k} $\in [0,1]$: proportion of \emph{current} inventory to sell at step $k$.
			\item Transformed to actual shares via $Q_{\mathrm{sell}}=a_k\,Q_{k}$.
			\item Alternative: \textbf{actions as trading rates} $u_k \in \mathbb{R}_+$.
		\end{itemize}
		\todo{Add equations linking agent action to price impact (see Eq.~(15) in \code{syntheticChrissAlmgren.py}).}
	\end{frame}
	
	% ================================================================
	\section{Reward Engineering}
	
	\begin{frame}{Custom Reward Function}
		\begin{block}{Objective}
			Minimise \textbf{Expected Shortfall (ES$_\alpha$)} at risk level $\alpha$ while penalising variance.
		\end{block}
		\vspace{-0.3cm}
		\[
		R = -\mathrm{ES}_{0.95}(\mathrm{P\&L}) - \lambda_1 \sigma^2 - \lambda_2\,\epsilon\,|Q_{\mathrm{sell}}|
		\]
		\begin{itemize}
			\item CVaR surrogate implemented via auxiliary variable $\eta$ (see \code{ddpg_agent.py}).
			\item Trading fee $\epsilon$ appended to imitate real markets.
			\item Discount factor $\beta=0.9999$ to emphasise early proceeds.
		\end{itemize}
		\vspace{0.5em}
		\textit{Result:} \todo{Insert comparison table: AC vs DDPG + custom reward.}
	\end{frame}
	
	\begin{frame}{Dense vs Sparse Rewards}
		\begin{itemize}
			\item \textbf{Dense}: agent rewarded at each step based on instantaneous proceeds.
			\item \textbf{Sparse}: zero reward until terminal liquidation.
		\end{itemize}
		\pause
		\textit{Observation}: Dense rewards accelerate convergence; sparse rewards encourage risk-aware behaviour but require longer training.
		\vspace{0.5em}
		\todo{Add learning-curve figure here.}
	\end{frame}
	
	% ================================================================
	\section{Enhanced Market Simulator}
	
	\begin{frame}{Price Dynamics – Switching to GBM}
		\begin{itemize}
			\item Replaced arithmetic Brownian motion with Geometric Brownian Motion (GBM):
			\[ dS_t = \mu S_t \mathrm{d}t + \sigma S_t \mathrm{d}W_t. \]
			\item Adjusted single-step variance in environment (see \code{syntheticChrissAlmgren.py}).
		\end{itemize}
		\vspace{0.5em}
		\textbf{Impact on Policy}
		\begin{itemize}
			\item Higher proportional variance $\Rightarrow$ larger tail-risk; CVaR term becomes more binding.
			\item Learned schedule skews toward \emph{front-loading} when drift $\mu<0$.
		\end{itemize}
		\todo{Insert P&L distribution plot (GBM vs ABM).}
	\end{frame}
	
	\begin{frame}{Adding Trading Fees \& Non-linear Impact}
		\begin{itemize}
			\item Fixed fee $\epsilon$ already in AC; we add proportional fee $\lambda_2 \times v^2$.
			\item Encourages smoother execution path.
		\end{itemize}
		\todo{Show before/after utility comparison.}
	\end{frame}
	
	% ================================================================
	\section{Algorithmic Benchmarks}
	
	\begin{frame}{DDPG Baseline (Our Implementation)}
		\begin{itemize}
			\item Actor–Critic with OU-noise exploration.
			\item Replay buffer size $10^4$, batch size 128.
			\item Achieved ES$_{0.95}$: \todo{XX\% improvement over AC.}
		\end{itemize}
	\end{frame}
	
	\begin{frame}{Space for Further Work}
		\begin{block}{SAC and TD3 Experiments}
			\begin{itemize}
				\item \todo{Insert architecture + hyper-parameters}
				\item \todo{Insert performance metrics vs DDPG}
			\end{itemize}
		\end{block}
		\vspace{0.5em}
		Please complete once models are trained and evaluated.
	\end{frame}
	
	% ================================================================
	\section{Conclusion}
	
	\begin{frame}{Key Takeaways}
		\begin{itemize}
			\item AC provides analytical insight but is rigid.
			\item Deep RL absorbs richer signals and objectives, outperforming AC on ES.
			\item Reward shaping and environment realism (GBM, fees) materially influence learned policy.
			\item Future: ensemble of SAC/TD3, calibration on LOB simulator (e.g. ABIDES).
		\end{itemize}
	\end{frame}
	
	% ----------------------------------------------------------------
	\begin{frame}[allowframebreaks]{References}
		\tiny
		\bibliographystyle{abbrv}
		\begin{thebibliography}{9}
			\bibitem{AC2001} R. Almgren and N. Chriss. \emph{Optimal Execution of Portfolio Transactions}. 2001.
			\bibitem{Gatheral2012} J. Gatheral and A. Schied. \emph{Optimal Trade Execution under GBM}. 2012.
			\bibitem{Cheridito2025} P. Cheridito and M. Weiss. \emph{Reinforcement Learning for Trade Execution with Market Impact}. arXiv:2507.06345, 2025.
			\bibitem{Hafsi2024} Y. Hafsi and E. Vittori. \emph{Optimal Execution with Reinforcement Learning}. arXiv:2411.06389, 2024.
		\end{thebibliography}
	\end{frame}
	
	% ================================================================
\end{document}
